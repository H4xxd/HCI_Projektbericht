\documentclass[12pt, a4paper]{scrartcl}

%used packages
\usepackage[utf8]{inputenc}
\usepackage[ngerman]{babel}
\usepackage{xcolor}
\usepackage{graphicx}
\usepackage{hyperref}
\usepackage{makecell}
\usepackage{sectsty}
\usepackage{blindtext}


\usepackage{fancyhdr}
\pagestyle{fancy}
\renewcommand{\sectionmark}[1]{\markright{#1}}
\renewcommand{\footrulewidth}{0.4pt}
\fancyhead{}
\fancyhead[R]{\rightmark}
\fancyfoot{}
\fancyfoot[R]{\thepage}


%costum commads
\newcommand{\mailto}[1]{\href{mailto:#1}{\color{silver}{#1}}}
\newcommand{\NameMail}[2]{{{#1}}\\{\small\mailto{#2}}}
\newcommand{\mysubsection}[1]{\subsection*{#1}
	\addcontentsline{toc}{subsection}{#1}}

%color scheme
\definecolor{spaceCadet}{HTML}{2D3142}
\definecolor{independence}{HTML}{4F5F75}
\definecolor{silver}{HTML}{BFC0C0}
\definecolor{tumbleweed}{HTML}{D7A28A}
\definecolor{mandarin}{HTML}{EF8354}
\definecolor{sinopia}{HTML}{D12911}

%hyperref setup
\hypersetup{
	colorlinks,
	linkcolor=sinopia,
	citecolor={blue!50!black},
	urlcolor={blue!80!black}
}

%editig section and subsection style
\sectionfont{\color{spaceCadet}}
\subsectionfont{\color{independence}}

\begin{document}
	\begin{titlepage}
		\raggedleft
		
		\textcolor{mandarin}{\rule{1pt}{\textheight}} 
		\hspace{0.05\textwidth}
		\parbox[b]{0.85\textwidth}{
			
			{\Huge\bfseries \textcolor{spaceCadet}{SmartMirror}}\\[1\baselineskip]
			{\LARGE \bfseries \textcolor{independence}{Fakultät Informatik Mathematik}}\\[1\baselineskip]
			{\Large  \textcolor{independence}{OTH Regensburg}}\\[1\baselineskip]
			{\large \textcolor{independence}{Human Computer Interaction}}\\[2\baselineskip]
			
			
			
			
			{\LARGE\bfseries\textcolor{spaceCadet}{Projektbericht}}\\[0.2\baselineskip]
			{\Large\textcolor{independence}{Wintersemester 2020/2021}} \\[0.2\baselineskip]
			{\textcolor{independence}{\noindent 12. Februar 2021}} \\[3\baselineskip]
			
			\vspace{0.4\textheight}
			\begin{tabular}{ c c }
				\makecell[l]{\NameMail{Patrick Gruber}{patrick.gruber@st.oth-regensburg.de}}
				& \makecell[l]{\NameMail{Tobias Gubo}{tobias1.gubo@st.oth-regensburg.de}}\\
				\makecell[l]{\NameMail{Michael Lazik}{michael1.lazik@st.oth-regensburg.de}} 
				& \makecell[l]{\NameMail{Marcus Müller}{marcus.mueller@st.oth-regensburg.de}}
			\end{tabular}\\ [2\baselineskip]
		}
		
	\end{titlepage}
	
	\tableofcontents
	\thispagestyle{empty}
	\pagebreak
	\setcounter{page}{1}
	
	\section{Planen des menschzentrierten Gestaltungsprozesses}
	\begin{quote}
		{Nur wer sein Ziel kennt, findet den Weg.} - Laozi
	\end{quote}
	Wie jedes gute Projekt beginnt man mit der grundlegenden Planung und Strukturierung 
	\subsection{Festlegen der Resourcen}
	Wie?\\
	Entsprechend dem Projekt (Zielgruppe, Inhalt \& Umfang) werden die einzusetzenden Ressourcen (Personen, Zeit, Materialien) bestimmt und kommuniziert\\
	Warum?\\
	Zeit-und Kostenplanung \& Einsatzplanungder vorhandenen Personen
	
	\newpage
	
	\section{Verstehen und Festlegen des Nutzungskontexts}
	\blindtext[2]
	\subsection{Beschreibung des Nutzungskontexts}
	\blindtext[1]
	
	\newpage
	
	\section{Festlegen der Nutzungsanforderungen}
	\blindtext[2]
	\subsection{Nutzungsanforderungen spezifizieren}
	\blindtext[1]
	\subsection{Nutzungsanforderungen priorisieren}
	\blindtext[1]
	
	\newpage
	
	\section{Erarbeitung von Gestaltungslösungen zur Erfüllung des Nutzungskontexts}
	\blindtext[2]
	\subsection{Erstellen der Interaktionsspezifikation}
	\blindtext[1]
	\subsection{Erstellen der Informationsarchitektur}
	\blindtext[1]
	\subsection{Spezifikation des UI}
	\blindtext[1]
	\subsection{Prototyping}
	\blindtext[1]
	\subsection{UI Gestaltungsrichtlinien}
	\blindtext[1]
	
	\newpage
	
	\section{Evaluierung der Gestaltungslösung anhand der Nutzungsanforderung}
	\blindtext[2]
	\subsection{Entwicklungsbegleitende Usability-Tests}
	\blindtext[1]
	\subsection{QuantitativeEvaluierungen}
	\blindtext[1]
	
	
	
\end{document}