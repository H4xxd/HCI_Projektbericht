\documentclass[12pt, a4paper]{scrartcl}

%used packages
\usepackage[utf8]{inputenc}
\usepackage[ngerman]{babel}
\usepackage{xcolor}
\usepackage{graphicx}
\usepackage{hyperref}
\usepackage{makecell}
\usepackage{sectsty}
\usepackage{blindtext}

\usepackage{tabularx}
\usepackage{eurosym}


\usepackage{fancyhdr}
\pagestyle{fancy}
\renewcommand{\sectionmark}[1]{\markright{#1}}
\renewcommand{\footrulewidth}{0.4pt}
\fancyhead{}
\fancyhead[R]{\rightmark}
\fancyfoot{}
\fancyfoot[R]{\thepage}


%costum commads
\newcommand{\mailto}[1]{\href{mailto:#1}{\color{silver}{#1}}}
\newcommand{\NameMail}[2]{{{#1}}\\{\small\mailto{#2}}}
\newcommand{\mysubsection}[1]{\subsection*{#1}
	\addcontentsline{toc}{subsection}{#1}}

%color scheme
\definecolor{spaceCadet}{HTML}{2D3142}
\definecolor{independence}{HTML}{4F5F75}
\definecolor{silver}{HTML}{BFC0C0}
\definecolor{tumbleweed}{HTML}{D7A28A}
\definecolor{mandarin}{HTML}{EF8354}
\definecolor{sinopia}{HTML}{D12911}

%hyperref setup
\hypersetup{
	colorlinks,
	linkcolor=sinopia,
	citecolor={blue!50!black},
	urlcolor={blue!80!black}
}

%editig section and subsection style
\sectionfont{\color{spaceCadet}}
\subsectionfont{\color{independence}}
\subsubsectionfont{\color{tumbleweed}}

\begin{document}
	\begin{titlepage}
		\raggedleft
		
		\textcolor{mandarin}{\rule{1pt}{\textheight}} 
		\hspace{0.05\textwidth}
		\parbox[b]{0.85\textwidth}{
			
			{\Huge\bfseries \textcolor{spaceCadet}{SmartMirror}}\\[1\baselineskip]
			{\LARGE \bfseries \textcolor{independence}{Fakultät Informatik Mathematik}}\\[1\baselineskip]
			{\Large  \textcolor{independence}{OTH Regensburg}}\\[1\baselineskip]
			{\large \textcolor{independence}{Human Computer Interaction}}\\[2\baselineskip]
			
			
			
			
			{\LARGE\bfseries\textcolor{spaceCadet}{Projektbericht}}\\[0.2\baselineskip]
			{\Large\textcolor{independence}{Wintersemester 2020/2021}} \\[0.2\baselineskip]
			{\textcolor{independence}{\noindent 12. Februar 2021}} \\[3\baselineskip]
			
			\vspace{0.4\textheight}
			\begin{tabular}{ c c }
				\makecell[l]{\NameMail{Patrick Gruber}{patrick.gruber@st.oth-regensburg.de}}
				& \makecell[l]{\NameMail{Tobias Gubo}{tobias1.gubo@st.oth-regensburg.de}}\\
				\makecell[l]{\NameMail{Michael Lazik}{michael1.lazik@st.oth-regensburg.de}} 
				& \makecell[l]{\NameMail{Marcus Müller}{marcus.mueller@st.oth-regensburg.de}}
			\end{tabular}\\ [2\baselineskip]
		}
		
	\end{titlepage}
	
	\tableofcontents
	\thispagestyle{empty}
	\pagebreak
	\setcounter{page}{1}
	
	\section{Planen des menschzentrierten Gestaltungsprozesses}
	\begin{quote}
		{Nur wer sein Ziel kennt, findet den Weg.} - Laozi
	\end{quote}
	Wie jedes gute Projekt beginnt man mit der grundlegenden Planung und Strukturierung 
	\subsection{Festlegen der Resourcen}
	Der erste und elemetarste Schritt war es sich zu überlegen, wie ein Produkt wie der SmartMirror überhaupt realisiert werden kann. Dafür haben wir uns im Internet auf die Suche gemacht und sind dort an mehreren Stellen auf DIY-Projekte gestoßen, die eine ähnliche Idee umgesetzt haben. Die fundamentalen Bestandteile waren jedoch oft sehr vergleichbar. Man benötigt einen Einwegspiegel, der das Licht von einer Seite durchlässt und von der anderen Seite verspiegelt ist. Mit einem LED-Display hinter der Spiegelscheibe lässt sich so eine Anzeigefläche erschaffen, die für den Betrachter nur sichtbar ist, wenn sie beleuchtet ist. Die Art und Weiße, wie das Display angesprochen wird ist wieder eine freiere Entscheidung. Wir haben uns dafür entschieden einen RaspberryPi 3b zu verwenden. Dies hat mehrere Gründe, die im Laufe des Berichts noch genauer beleuchtet werden.Montiert wird das Spiegeldisplay in einem Rahmen aus Holz, um ein einheitliches Erscheinungsbild zu kreieren. Als letztes wichtiges Element ist noch der LeapMotion-Controller zu nennen, der eine einfache Möglichkeit bildet um die Gestik des Nutzers zu erkennen, um somit die Interaktion zwischen Mensch und Maschine zu ermöglichen.
	
	\subsubsection*{Grobe Kostenaufstellung}
	\begin{tabularx}{0.95\textwidth}{|X|l|r|}
		\hline
		\textcolor{tumbleweed}{\underline{\textbf{Name}}} & \textcolor{tumbleweed}{\underline{\textbf{Zulieferer}}} & \textcolor{tumbleweed}{\underline{\textbf{Preis}}}\\
		\hline
		Spiegelglas 70x100cm & GlasStar&ca. 125\euro\\
		\hline
		Display 17" mit Controller & Amazon & ca. 160\euro\\
		\hline
		LeapMotion Controller & AdaFruit & ca. 100\euro \\
		\hline
		Holzrahmen 70x100cm & Amazon & ca. 50\euro\\
		\hline
		Raspberry Pi & Amazon & ca. 40\euro\\
		\hline
		Kabel und Verbidungen & Amazin & ca. 50\euro\\
		\hline
		\textcolor{tumbleweed}{\textbf{Gesamt}}& & \textbf{ca. 525\euro}\\
		\hline
	\end{tabularx}
	
	\subsection*{Aufgabenaufteilung}
	\begin{tabularx}{0.95\textwidth}{|l|X|}
		\hline
		Patrick Gruber & Erstellen der Kostenaufstellung\\
		\hline
		Tobias Gubo & \\
		\hline
		Michael Lazik & \\
		\hline
		Marcus Müller & \\
		\hline
	\end{tabularx}
	
	\newpage
	
	\section{Verstehen und Festlegen des Nutzungskontexts}
	Nachdem die Ressourcen festgelegt worden sind, musste der Nutzungskontext definiert, verstanden und festgelegt werden. Hierbei wurde überlegt, wer die Benutzer sind und wie die Benutzer mit dem Smart Mirror interagieren werden.
	\subsection{Beschreibung des Nutzungskontexts}
	Unsere Primären (Direkten) Benutzer sind die Probanden, die unser Smart Mirror testen und Fazite daraus ziehen.\\
	Die Sekundären (Indirekten) Benutzer sind die Mitglieder des Projektteams, die aus den Fazite den Smart Mirror weiter entwickeln.\\
	Mit dem Smart Mirror sollen den Benutzern innerhalb kürzester Zeit die wichtigsten Informationen am Spiegel gezeigt werden.\\
	Dabei sollen dem Benutzer gezeigt werden:
	\begin{itemize}
		\setlength\itemsep{-0.5em}
		\item Uhrzeit
		\item Wetter (Uhrzeitbedingt)
		\item RSS-Feed
	\end{itemize}
	Zudem kann der Benutzer seine individuellen Informationen anzeigen lassen. In unserem Projekt sind das:
	\begin{itemize}
		\setlength\itemsep{-0.5em}
		\item Termine
		\item  ToDo-Liste
		\item Fahrzeiten ÖPNV
		\item Verkehrslage
		\item Speißeplan
	\end{itemize}
	Für den Smart Mirror benötigt der Nutzer keine Ausrüstung. Sie müssen nur wissen, wie man mit dem Smart Mirror interagiert und wie man durch den Pageflow die Seiten wechselt. Der Smart Mirror kann an jeden von den Benutzern beliebig ausgewählten Ort stehen. Das Projektteam hat sich jedoch auf das Schlafzimmer als Örtlichkeit für den Smart Mirror festgelegt.\\
	
	Damit der Nutzungskontext aus Sicht der primären Benutzer verstanden ist, wurden Interviews mit den Probanden geführt.
	Dabei wurden vor allem zu Beginn möglichst offene, neutrale und allgemeine Fragen gestellt, also das "Warum?", die dann immer mehr in die Materie des Smart Mirrors gingen, also das "Wie?". Hier konnten die Probanden aus ihre Vorstellung eines Smart Mirrors offenbaren und das Projektteam konnte anhand dieser Informationen die Nutzungsanforderungen festlegen. 
	

	
	
	\subsection*{Aufgabenaufteilung}
	\begin{tabularx}{0.95\textwidth}{|l|X|}
		\hline
		Patrick Gruber & \\
		\hline
		Tobias Gubo & \\
		\hline
		Michael Lazik & \\
		\hline
		Marcus Müller & \\
		\hline
	\end{tabularx}
	
	\newpage
	
	\section{Festlegen der Nutzungsanforderungen}
	\blindtext[1]
	\subsection{Nutzungsanforderungen spezifizieren}
	Wie?\\
	Auswertung der ermittelten Erfordernisse in
	Hinblick auf nötige Systemunterstützung
	aus Nutzersicht.\\
	Warum?\\
	Nutzungsanforderungen beschreiben nicht die Lösung, sondern die konkreten Handlungsmöglichkeiten des Nutzers, die durch die Lösung unterstützt werden müssen. Ohne Nutzungsanforderungen sind erarbeitete Lösungen nicht validierbar.
	\subsection{Nutzungsanforderungen priorisieren}
	Wie?\\
	Die erhobenen Nutzungsanforderungen
	werden im Projektteam nach Relevanz für den Nutzer und Umsetzbarkeit unter Kosten- / Nutzenaspekten priorisiert.\\
	Warum?\\
	Eine Priorisierung von Nutzungsanforderungen ist erforderlich, um Produktversionen gezielt unter wirtschaftlichen Aspekten planen zu können.
	
	\subsection*{Aufgabenaufteilung}
	\begin{tabularx}{0.95\textwidth}{|l|X|}
		\hline
		Patrick Gruber & \\
		\hline
		Tobias Gubo & \\
		\hline
		Michael Lazik & \\
		\hline
		Marcus Müller & \\
		\hline
	\end{tabularx}
	
	\newpage
	
	\section{Erarbeitung von Gestaltungslösungen zur Erfüllung des Nutzungskontexts}
	\blindtext[2]
	\subsection{Erstellen der Interaktionsspezifikation}
	\blindtext[1]
	\subsection{Erstellen der Informationsarchitektur}
	\blindtext[1]
	\subsection{Spezifikation des UI}
	\blindtext[1]
	\subsection{Prototyping}
	\blindtext[1]
	\subsection{UI Gestaltungsrichtlinien}
	\blindtext[1]
	
	\subsection*{Aufgabenaufteilung}
	\begin{tabularx}{0.95\textwidth}{|l|X|}
		\hline
		Patrick Gruber & \\
		\hline
		Tobias Gubo & \\
		\hline
		Michael Lazik & \\
		\hline
		Marcus Müller & \\
		\hline
	\end{tabularx}
	
	\newpage
	
	\section{Evaluierung der Gestaltungslösung anhand der Nutzungsanforderung}
	\blindtext[2]
	\subsection{Entwicklungsbegleitende Usability-Tests}
	\blindtext[1]
	\subsection{QuantitativeEvaluierungen}
	\blindtext[1]
	
	\subsection*{Aufgabenaufteilung}
	\begin{tabularx}{0.95\textwidth}{|l|X|}
		\hline
		Patrick Gruber & \\
		\hline
		Tobias Gubo & \\
		\hline
		Michael Lazik & \\
		\hline
		Marcus Müller & \\
		\hline
	\end{tabularx}
	
	
	
\end{document}